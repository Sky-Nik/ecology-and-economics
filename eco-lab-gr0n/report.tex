% cd ..\..\Users\NikitaSkybytskyi\Desktop\eco-lab
% cls && pdflatex report.tex && cls && pdflatex report.tex && del report.aux, report.toc, report.log, report.out && start report.pdf
\input{lab.sty}

\begin{document}

\cover{1}{Поведінка ізольованої популяції. \\ Модель Леслі вікової структури}

\tableofcontents

\section{Поведінка ізольованої популяції}

Два зоопарки вирішили вирощувати хом'яків на продаж. Кількість хом'яків $P(t)$ ($t$ виражається в місяцях) задовольняє диференціальне рівняння \[ \frac{\diff P}{\diff t} = 0.05 \cdot P(t) \cdot (145 - P(t)). \] Перший зоопарки закупив 195 хом'яків, другий -- 90. \\

Розв'яжіть диференціальне рівняння та визначте, що станеться з популяцією хом'яків? \\

Побудувати графіки чисельності популяцій для двох випадків.

\subsection{Аналітичний розв'язок}

Запишемо обидва питання як задачі Коші для рівняння Бернуллі. \\

Перша задача:
\begin{equation*}
	\left\{
		\begin{aligned}
			& P'(t) = 0.05 \cdot P(t) \cdot (145 - P(t)), \\
			& P(0) = 195.
		\end{aligned}
	\right.
\end{equation*}

Друга задача:
\begin{equation*}
	\left\{
		\begin{aligned}
			& P'(t) = 0.05 \cdot P(t) \cdot (145 - P(t)), \\
			& P(0) = 90.
		\end{aligned}
	\right.
\end{equation*}

Як відомо з курсу диференціальних рівнянь, загальним розв'язком цього рівняння Бернуллі є функція
\begin{equation*}
	P(t) = \frac{145 \cdot e^{7.25 t}}{e^{7.25 t} + C},
\end{equation*}
де стала $C$ визначається початковими умовами. \\

Для першої задачі Коші $C = - \frac{10}{39}$, і відповідний розв'язок набуває вигляду:
\begin{equation*}
	P(t) = \frac{145 \cdot e^{7.25 t}}{e^{7.25 t} - \frac{10}{39}},
\end{equation*}

Для другої задачі Коші $C = \frac{11}{18}$, і відповідний розв'язок набуває вигляду:
\begin{equation*}
	P(t) = \frac{150 \cdot e^{7.25 t}}{e^{7.25 t}  + \frac{11}{18}}.
\end{equation*}

Графіки цих аналітичних розв'язків:
\begin{figure}[H]
	\centering
	\includegraphics[width=\textwidth]{1_analytical.png}
\end{figure}

\subsection{Чисельне моделювання}

Було використано мову програмування \texttt{Python} і модуль \texttt{scipy.integrate}. Лістинг коду програми:
\inputminted{python}{1_numerical.py}

Графіки отриманих чисельних розв'язків:
\begin{figure}[H]
	\centering
	\includegraphics[width=\textwidth]{1_numerical.png}
\end{figure}

Як бачимо отримані графіки (майже) не відрізняються.

\subsection{Висновки про задачу}

Модель що розглядається в задачі -- модель Верхюльста, канонічним записом якої є 
\begin{equation*}
    \frac{\diff N}{\diff t} = r \cdot N \cdot \left( 1 - \frac{N}{q} \right), \quad N(0) = N_0,
\end{equation*}
де $r$ -- коефіцієнт росту. Іншою назвою є ``рівняння логістичного росту''. У нашій задачі коефіцієнт росту -- $145 \cdot 0.05 = 7.25$. \\

Як відомо з теоретичного курсу лекції, модель враховує внутрішньовидову конкуренцію (містить член $-c \cdot N^2$), тому популяції не зростають необмежено а прямують до ємності середовища, у нашому випадку -- 145 особин хом'яків. \\

Отримані аналітичні та чисельні розв'язки повністю підтверджують теоретичні результати.

\section{Модель Леслі вікової структури}

Вихідна популяція складається з трьох вікових груп. \\

Матриця Леслі має вигляд \[ L = \begin{pmatrix} 0 & 9 & 15 \\ 1/3 & 0 & 0 \\ 0 & 1/2 & 0 \end{pmatrix} \] У початковий момент часу ($t = 0$) популяція складається з однієї самки кожмолодшого та однієї самки старшого віку. \\

Знайти:
\begin{enumerate}
	\item склад $x(t)$ у момент часу $t = 10$;
	\item стійку вікову структуру популяції;
	\item момент часу, коли загальна кількість популяції перевищить 90 особин.
\end{enumerate}

\subsection{Чисельне моделювання}

Було використано мову програмування \texttt{Python} і модуль \texttt{numpy.linalg}. Лістинг коду програми:
\inputminted{python}{2.py}

Запишемо $x(t) = L^t \cdot x(0)$, звідси \[ x(10) = L^{10} \cdot x(0) = (4864.875, 781.45833333, 191.625)^T. \]

Для визначення стійкої структури популяції знайдемо власні числа матриці Леслі: \[ \det (L - \lambda E) = \begin{vmatrix} -\lambda & 9 & 15 \\ 1 / 3 & -\lambda & 0 \\ 0 & 1 / 2 & -\lambda \end{vmatrix} = - \lambda^3 + 3 \lambda + \frac{5}{2} = 0. \]

Найбільшим додатним власним числом є \[ \lambda_L = \frac{1}{2} \cdot \left( 2 \cdot \sqrt[3]{2} + \sqrt[3]{4} \right) \approx 2.05362. \]

Цьому власному числу відповідає власний вектор \begin{align*} x_L &= \left( 3 \cdot \left( 4 + \sqrt[3]{2} + 2 \cdot \sqrt[3]{4} \right), 2 \cdot \sqrt[3]{2} + \sqrt[3]{4}, 1 \right)^T \approx \\ &\approx (18.9006, 4.34748, 1)^T. \end{align*}

Для зручності пронормуємо цей вектор у $\|\cdot\|_1$-нормі, щоб дізнатися відсоткове співвідношення чисельностей вікових груп, отримаємо \[ \tilde x_L = (0.83206163 0.13505598 0.03288239)^T. \]

Як бачимо, у стійкій віковій структурі популяції 83.2\% особин першої вікової категорії, 13.5\% особин другої вікової категорії, і 3.2\% особин третьої вікової категорії. \\

Останнє завдання розв'язуємо простим циклом, ось його вивід:
\begin{table}[H]
	\centering
	\begin{tabular}{|c|c|} \hline
		$t$ & $\|x(t)\|_1$ \\ \hline
		0 & 2 \\ \hline
		1 & 15.333333333333334 \\ \hline
		2 & 8.166666666666666 \\ \hline
		3 & 51.0 \\ \hline
		4 & 62.83333333333333 \\ \hline
		5 & 173.41666666666666 \\ \hline
	\end{tabular}
\end{table}

Як бачимо, чисельність популяції вперше перевищує 90 особин у момент часу $t = 5$.

\end{document}