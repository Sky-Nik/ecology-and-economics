\documentclass[a4paper, 12pt]{article}
\usepackage[utf8]{inputenc}
\usepackage[T1, T2A]{fontenc}
\usepackage[english, ukrainian]{babel}

\usepackage{amsmath, amssymb}
\usepackage{multicol}
\usepackage{graphicx}
\usepackage{float}

\allowdisplaybreaks
\setlength\parindent{0pt}
\numberwithin{equation}{subsection}

\usepackage[unicode=true,colorlinks=true,linktoc=all,linkcolor=blue, pdfencoding=auto]{hyperref}
\usepackage{bookmark}

\numberwithin{equation}{subsection}

\renewcommand{\bf}[1]{\textbf{#1}}
\renewcommand{\it}[1]{\textit{#1}}
\newcommand{\bb}[1]{\mathbb{#1}}
\renewcommand{\cal}[1]{\mathcal{#1}}

\renewcommand{\epsilon}{\varepsilon}
\renewcommand{\phi}{\varphi}

\DeclareMathOperator{\grad}{grad}
\DeclareMathOperator{\divergence}{div}
\DeclareMathOperator{\rot}{rot}

\DeclareMathOperator{\diam}{diam}
\DeclareMathOperator{\real}{Re}
\DeclareMathOperator{\rang}{rang}
\DeclareMathOperator{\supp}{supp}
\DeclareMathOperator{\const}{const}

\newenvironment{system}{%
  \begin{equation}%
    \left\{%
      \begin{aligned}%
}{%
      \end{aligned}%
    \right.%
  \end{equation}%
}
\newenvironment{system*}{%
  \begin{equation*}%
    \left\{%
      \begin{aligned}%
}{%
      \end{aligned}%
    \right.%
  \end{equation*}%
}

\newenvironment{nalign}{%
  \begin{equation}%
      \begin{aligned}%
}{%
      \end{aligned}%
  \end{equation}%
}
\newenvironment{nalign*}{%
  \begin{equation*}%
      \begin{aligned}%
}{%
      \end{aligned}%
  \end{equation*}%
}

\makeatletter
\newcommand*{\relrelbarsep}{.386ex}
\newcommand*{\relrelbar}{%
  \mathrel{%
    \mathpalette\@relrelbar\relrelbarsep%
  }%
}
\newcommand*{\@relrelbar}[2]{%
  \raise#2\hbox to 0pt{$\m@th#1\relbar$\hss}%
  \lower#2\hbox{$\m@th#1\relbar$}%
}
\providecommand*{\rightrightarrowsfill@}{%
  \arrowfill@\relrelbar\relrelbar\rightrightarrows%
}
\providecommand*{\leftleftarrowsfill@}{%
  \arrowfill@\leftleftarrows\relrelbar\relrelbar%
}
\providecommand*{\xrightrightarrows}[2][]{%
  \ext@arrow 0359\rightrightarrowsfill@{#1}{#2}%
}
\providecommand*{\xleftleftarrows}[2][]{%
  \ext@arrow 3095\leftleftarrowsfill@{#1}{#2}%
}
\makeatother

\newcommand{\NN}{\mathbb{N}}
\newcommand{\ZZ}{\mathbb{Z}}
\newcommand{\QQ}{\mathbb{Q}}
\newcommand{\RR}{\mathbb{R}}
\newcommand{\CC}{\mathbb{C}}

\newcommand{\Max}{\displaystyle\max\limits}
\newcommand{\Sup}{\displaystyle\sup\limits}
\newcommand{\Sum}{\displaystyle\sum\limits}
\newcommand{\Int}{\displaystyle\int\limits}
\newcommand{\Oint}{\displaystyle\oint\limits}
\newcommand{\Iint}{\displaystyle\iint\limits}
\newcommand{\Iiint}{\displaystyle\iiint\limits}
\newcommand{\Lim}{\displaystyle\lim\limits}

\newcommand{\Bigcup}{\displaystyle\bigcup\limits}
\newcommand{\Bigsqcup}{\displaystyle\bigsqcup\limits}
\newcommand{\Bigcap}{\displaystyle\bigcap\limits}
\newcommand{\Bigsqcap}{\displaystyle\bigsqcap\limits}

\usepackage{esint}
\newcommand{\Oiint}{\displaystyle\oiint\limits}
\newcommand{\Oiiint}{\displaystyle\oiiint\limits}

\newcommand*\diff{\mathop{}\!\mathrm{d}}

\newcommand*\rfrac[2]{{}^{#1}\!/_{\!#2}}

\usepackage{array}
\newcolumntype{L}[1]{>{\raggedright\let\newline\\\arraybackslash\hspace{0pt}}m{#1}}
\newcolumntype{C}[1]{>{\centering\let\newline\\\arraybackslash\hspace{0pt}}m{#1}}
\newcolumntype{R}[1]{>{\raggedleft\let\newline\\\arraybackslash\hspace{0pt}}m{#1}}

\newcommand{\oG}{\overline{G}}
\newcommand{\CoG}{C\left(\oG\right)}
\newcommand{\CoGoG}{C\left(\oG\times\oG\right)}

\usepackage{bm}

\newcommand{\vecf}[1]{{\vec{\mathbf{#1}}}} % vector-field or vector-function

\let\oldlangle=\langle
\renewcommand{\langle}{\left\oldlangle}
\let\oldrangle=\rangle
\renewcommand{\rangle}{\right\oldrangle}

% _s stands for "safe"
\makeatletter
\newcommand{\seqref}[1]{%
  \@ifundefined{r@eq:#1}{%
    (\textcolor{blue}{#1})%
  }{%
    \eqref{eq:#1}%
  }%
}
\makeatother

\author{Нікіта Скибицький}
\title{Екологічні та економічні процеси, та їхнє моделювання}
\date{\today}

\begin{document}

\maketitle

\tableofcontents

\section*{Логістика}

Лектор Колянова Тетяна Володимирівна, з кафедри МІ, знайти можна у аудиторії 711 на сьомому поверху, \texttt{tania.kolianova@gmail.com}. \\

Остання лекція 15-го травня, практика 24-го квітня чи десь там, мабуть там і проведемо залік.

\section*{Екологічні процеси}

В принципі досить складні, бо живі істоти що входять до моделі самі по собі дуже складні. Зачасту для моделювання застосовуються диференціальні рівняння або системи диференціальних рівнянь. \\

Екологічні процеси дозволяють пояснити зміст фазових портретів системи, стійких і нестійких точок і так далі, тобто надати диференціальним рівнянням певного сенсу.

\section{Моделі динаміки ізольованої популяції}

\subsection{Найпростіші математичні моделі динаміки популяції}

Моделі бувають неперервні і дискретні.

\subsubsection{Неперервні моделі}

Літерою $N$ будемо позначати чисельність (густина, щільність, обсяг, об'єм, кількість) популяції. \\

Найпростіша модель росту популяції організмів задається рівнянням Бернуллі (1760 р.)
\begin{equation}
    \frac{\diff N}{\diff t} = \mu \cdot N, \quad N(0) = N_0,
\end{equation}
де $t$ -- час, і $\mu = B - D$, різниця між народжуваністю $B$ і смертністю $D$. \\

Розв'язком цього рівняння
\begin{equation}
    N(t) = N_0 \cdot e^{\mu t}.
\end{equation}

При $\mu = const > 0$ маємо експоненціальний ріст, більш відомий як закон Мальтуса, або закон експоненціального росту популяції в необмеженому середовищі (в розумінні поживності). \\

Якщо $\mu > 0$ і $t \to + \infty$ маємо $N(t) \to + \infty$. Якщо ж $\mu < 0$ то при $t \to +\infty$ маємо $N(t) \to 0$. \\

Першим рівнянням описується доволі багато процесів, таких як радіоактивний розпад або ріст дріжджів, але він неможливий на нескінченному проміжку. \\

В природі для багатьох популяцій при $t \to + \infty$ маємо $N(t) \to K = const$ (ємність середовища). Цим умовам відповідає, наприклад, модель Гемпертца (1825 р.):
\begin{equation}
    \frac{\diff N}{\diff t} = - \mu \cdot \frac{N \cdot \ln \frac NK}{\ln K}, \quad N(0) = N_0.
\end{equation}

Розв'язок цього рівняння має вигляд
\begin{equation}
    N(t) = K \cdot \left( \frac{N_0}{K} \right)^{e^{- \frac{\mu}{\ln K}}}.
\end{equation}

Зауважимо, що цією моделюю можна користуватися тільки якщо відомий параметр $K$. \\

Наступна модель Верхюльста (1838 р.):
\begin{equation}
    \frac{\diff N}{\diff t} = r \cdot N \cdot \left( 1 - \frac{N}{q} \right), \quad N(0) = N_0,
\end{equation}
де $r$ -- коефіцієнт росту, $q = K$. Іншою назвою є ``рівняння логістичного росту''. \\

Це одне з найперших рівнянь яке враховує внутрішньовидову конкуренцію (містить член $-c \cdot N^2$). \\

Розв'язком цього рівняння є
\begin{equation}
    N(t) = \frac{q}{1 - \left( 1 - \frac{q}{N_0} \right) \cdot e^{-rt}}. 
\end{equation}

При $t \to + \infty$ маємо $N(t) \to q$. \\

Це певні основні моделі які часто використовуються. Взагалі кажучи різних моделей набагато більше, але це основні з якими ми будемо працювати багато на лабораторних заняттях.

\subsubsection{Дискретні моделі}

Замінимо і останній моделі похідну її дискретним аналогом:

\begin{equation}
    N_{i + 1} \approx N_i \cdot \exp \left( r - \left( 1 - \frac{N_i}{q} \right) \right).
\end{equation}

Ця модель носить назву ``модель Ріккера''. \\

Іншими вченими була запропонована ще ось така дискретна модель:
\begin{equation}
    N_{i + 1} \approx \frac{\lambda \cdot N_i}{(1 + a \cdot N_i)^b},
\end{equation}
де $\lambda$, $a$, $b$ -- деякі параметри системи. Ця модель може бути застосованою до багатьох популяцій. \\

У найзагальнішій формі дискретна модель має вигляд
\begin{equation}
    N_{i + 1} = N_i \cdot F(N_i),
\end{equation}
де $F(\cdot)$ -- певна функція.

\subsection{Модель Леслі вікової структури}

У деяких популяціях врахування вікової структури має досить істотне значення. У життєвому циклі будь-якого організму можна виділити кілька стадій розвитку або вікових сходинок. \\

Розглядаючи далі деяку популяцію вважатимемо, що вона складатиметься з $n$ вікових груп. Спосіб розбиття на вікові групи визначається, як правило, біологічними особливостями організмів та специфікою задачі. \\

Вікові структури мають різну ймовірність виживання та народжуваності для кожного періоду. Найпростіша модель яка враховую віковий ценз та ймовірність виживання це модель Леслі (1945 р.): \\

Нехай $x_i(t)$ -- чисельність $i$-ої вікової групи (якщо не враховувати поділ на статі), $i = \overline{1, n}$. Якщо ж поділ на статі істотний то $x_i(t)$ -- чисельність самок $i$-ої вікової групи. \\

Змінна $t$ враховує лише дискретні зміни часу при переході від однієї вікової групи до наступної. \\

Для зручності вводиться вектор стану вікової структури 
\begin{equation}
    X = \left(x_1, x_2, \ldots, x_n\right)^T.
\end{equation}

Вважатимемо, що функція народжуваності $b(x)$ та функції, що характеризують перехід від однієї вікової структури до іншої $s(x)$ мають вигляд 
\begin{align}
    b_i(x) &= b_i \cdot x_i \quad (i = \overline{1, n}), \\
    s_i(x) &= s_i \cdot x_i \quad (0 \le s_i \le 1).
\end{align}

Тоді чисельність кожної із вікових груп можна описати наступними співвідношеннями
\begin{align}
    x_1(t + 1) &= \sum_{i = 1}^n c_i \cdot x_i(t), \\
    x_{i + 1}(t) &= x_i \cdot x_i(t), \quad i = \overline{i, n - 2}, \\
    x_n(t + 1) &= s_{n - 1} \cdot x_{n - 1}(t) + s_n \cdot x_n(t).
\end{align}

Введемо матрицю Леслі:
\begin{equation}
    L = \begin{pmatrix} b_1 & b_2 & \cdots & b_{n - 1} & b_n \\ s_1 & 0 & \cdots & 0 & 0 \\ 0 & s_2 & \cdots & 0 & 0 \\ \vdots & \vdots & \ddots & \vdots & \vdots \\ 0 & 0 & \cdots & s_{n - 1} & s_n \end{pmatrix}
\end{equation}

Тоді систему Леслі можна записати у матричному вигляді:
\begin{equation}
    X(t + 1) = L \cdot X(t).
\end{equation}

Якщо відомий початковий розподіл популяції $X(0) = X_0$, то маємо 
\begin{equation}
    X(t + 1) = L^{t + 1} \cdot X(0).
\end{equation}

Оскільки $b_i, s_i \ge 0$, то $L$ невід'ємно-визначена, тому для неї виконується теорема Перона-Фробеніуса, тобто знайдеться $\lambda_L$ -- найбільше додатне власне число (число Фробеніуса, швидкість росту). \\

Тоді власний вектор $X_L = \left( x_1^L, \ldots, x_n^L \right)$ що відповідає $\lambda_L$ -- вектор стійкої вікової структури популяції, тобто пропорція між чисельностями вікових груп збігається до пропорції у цьому векторі.

\begin{example*}
    Побудувати модель Леслі з трьох вікових груп. Знайти швидкість росту та стійку вікову структуру популяції.
\end{example*}

\begin{solution}
    Будуємо матрицю Леслі:
    \[ L = \begin{pmatrix} 0 & 9 & 12 \\ \frac{1}{3} & 0 & 0 \\ 0 & \frac{1}{2} & 0 \end{pmatrix}. \]
    Знаходимо його власні значення з характеристичного рівняння:
    \[ \begin{vmatrix} - \lambda & 9 & 12 \\ \frac{1}{3} & - \lambda & 0 \\ 0 & \frac{1}{2} & - \lambda \end{vmatrix} = - \lambda^3 + 3 \lambda + 2 = ( \lambda + 1)^2 \cdot (\lambda - 2) = 0. \]
    Таким чином маємо:
    \[ \lambda_L = 2, \quad X_L = \left( 24, 4, 1 \right)^T. \]
\end{solution}

\section{Моделювання взаємодії біологічних видів}

\subsection{Модель ``хижак-жертва''}

Стосунки з умовною характеристикою ``хижак-жертва'' є найбільш істотними для функціонування екосистеми. В основу відповідної моделі покладені ідеалізовані уявлення про характер внутрішньо- та між-видових стосунків у спільноті що складається з виду ``хижак'' і виду ``жертва'':
\begin{enumerate}
    \item За умови відсутності хижака популяція жертви розмножується експоненціально.
    \item За відсутності жертви популяція хижака експоненціально вимирає.
    \item Сумарна кількість біомаси жертв, що споживається популяцією хижака за одиницю часу лінійно залежить від густини популяції жертви та від щільності популяції хижака.
    \item Біомаса жертви, що споживається популяцією хижака перетворюється з певним коефіцієнтом у біомасу хижака.
    \item Будь-які додаткові фактори, що впливають на динаміку популяцій жертви та хижака відсутні.
\end{enumerate}

За цих припущень дана модель може бути описана у вигляді:
\begin{equation}
    \label{eq:2.1}
    \left\{
\begin{aligned}
    \frac{\diff x}{\diff t} &= \epsilon_1 x - \gamma_1 x y, \\
    \frac{\diff y}{\diff t} &= - \epsilon_2 y + \gamma_2 x y, 
\end{aligned}
\right.
\end{equation}
де $x$ -- густина популяції жертви, $y$ -- густина популяції хижака, $\epsilon_1 > 0$ -- швидкість розмноження жертви за відсутності хижака, $\epsilon_2 > 0$ -- швидкість загибелі хижака за відсутності жертви, $\gamma_1 > 0$ -- питома швидкість споживання популяцією хижака популяції жертви за одиничної густоти обох популяцій, а $\gamma_2 > 0$ -- коефіцієнт перетворення біомаси жертви, що була спожита хижаком на його біомасу. \\

Для знаходження стаціонарних станів системи \eqref{eq:2.1} розв'язуємо СЛАР
\begin{equation}
    \left\{
    \begin{aligned}
    \epsilon_1 x - \gamma_1 x y &= 0, \\
    \epsilon_1 x - \gamma_1 x y &= 0.
\end{aligned}
\right.
\end{equation}

Маємо дві точки:
\begin{enumerate}
    \item $x^* = y^* = 0$, тривіальна.
    \item $x^* = \epsilon_2 / \gamma_2$, $y^* = \epsilon_1 / \gamma_1$, не тривіальна.
\end{enumerate}

Проаналізуємо знайдені стаціонарні стани на стійкість. Для цього лінеаризуємо в околі кожної точки нашу систему: в загальному випадку в околі $(x_0, y_0)$:
\begin{equation}
    \left\{
    \begin{aligned}
    \frac{\diff x}{\diff t} = (\epsilon_1 - \gamma_1 y_0) \cdot (x - x_0) - \gamma_1  x_0 \cdot (y - y_0), \\
    \frac{\diff y}{\diff t} = \gamma_2  x_0 \cdot (y - y_0) - (\epsilon_2 - \gamma_2 y_0) \cdot (x - x_0),
\end{aligned}
\right.
\end{equation}

Для $(0,0)$ маємо $\lambda_1 = \epsilon_1$, $\lambda_2 = - \epsilon_2$, сідло. \\

Для $(\epsilon_2 / \gamma_2, \epsilon_1 / \gamma_1)$ маємо 

\begin{equation}
    \left\{
    \begin{aligned}
    \frac{\diff x}{\diff t} = - \frac{\gamma_1 \epsilon_2}{\gamma_2} \cdot \left(y - \frac{\epsilon_1}{\gamma_1} \right), \\
    \frac{\diff y}{\diff t} = \frac{\gamma_2 \epsilon_1}{\gamma_1} \cdot \left(x - \frac{\epsilon_2}{\gamma_2} \right).
\end{aligned}
\right.
\end{equation}

Робимо заміну змінних $x - \epsilon_2 / \gamma_2 = u$, $y - \epsilon_1 / \gamma_1 = v$, тоді
\begin{equation}
    \left\{
    \begin{aligned}
    \frac{\diff u}{\diff t} = - \frac{\gamma_1 \epsilon_2}{\gamma_2} \cdot v, \\
    \frac{\diff v}{\diff t} = \frac{\gamma_2 \epsilon_1}{\gamma_1} \cdot u,
\end{aligned}
\right.
\end{equation}

Знаходимо характеристичний многочлен як визначника і власні числа як його корені:

\begin{equation}
    \begin{vmatrix} - \lambda & - \frac{\gamma_1 \epsilon_2}{\gamma_2} \\ \frac{\gamma_2 \epsilon_1}{\gamma_1} & - \lambda \end{vmatrix} = \lambda^2 + \frac{\gamma_1 \epsilon_2}{\gamma_2} \cdot \frac{\gamma_2 \epsilon_1}{\gamma_1} = \lambda^2 + \epsilon_1 \epsilon_2,
\end{equation}

звідки $\lambda_{1, 2} = \pm i \sqrt{\epsilon_1 \epsilon_2}$ -- центр, бо $Re \lambda = 0$. \\

Нескладно знайти, що рух довкола центру буде відбуватися за еліпсами. \\

Модель ``хижак-жертва'' ще носить назву модель Вольтерра.

\subsection{Модель ``хижак-жертва'' з врахування внутрішньовидової конкуренції}

Ця модель носить назву модель Лоткі-Вольтерра. Її модель має наступний вигляд:

\begin{equation}
    \label{eq:2.2}
    \left\{
\begin{aligned}
    \frac{\diff x}{\diff t} &= \epsilon_1 x - \gamma_1 x y - \beta_1 x^2, \\
    \frac{\diff y}{\diff t} &= - \epsilon_2 y + \gamma_2 x y, 
\end{aligned}
\right.
\end{equation}

Ця корекція покликана виправити експоненційний ріст. \\

Проаналізуємо знайдені стаціонарні стани на стійкість:
\begin{equation}
    \left\{
    \begin{aligned}
    & x \cdot (\epsilon_1 - \gamma_1 y - \beta_1 x) = 0, \\
    & y \cdot (- \epsilon_2 + \gamma_2 x) = 0,
\end{aligned}
\right.
\end{equation}

маємо три стаціонарні точки:
\begin{enumerate}
    \item $x^* = y^* = 0$, тривіальна.
    \item $x^* = \epsilon_1 / \beta_1$, $y^* = 0$, напівтривіальна.
    \item $x^* = \frac{\epsilon_2}{\gamma_2}$, $\gamma^* = \dfrac{\epsilon_1 - \gamma_2 - \beta_1 \epsilon_2}{\gamma_1 \gamma_2} > 0$, або 
    \begin{equation}
        \label{eq:2.3}
        \frac{\epsilon_1}{\epsilon_2} > \frac{\beta_1}{\gamma_2}.
    \end{equation} 
\end{enumerate}

Знову-ж таки, лінеаризована система в околі точки $(x_0, y_0)$ матиме вигляд:
\begin{equation}
    \left\{
    \begin{aligned}
    \frac{\diff x}{\diff t} = (\epsilon_1 - \gamma_1 y_0 - 2 \beta_1 x_0) \cdot (x - x_0) - \gamma_1  x_0 \cdot (y - y_0), \\
    \frac{\diff y}{\diff t} = \gamma_2  x_0 \cdot (y - y_0) - (\epsilon_2 - \gamma_2 y_0) \cdot (x - x_0),
\end{aligned}
\right.
\end{equation}

Для точки $(0, 0)$:
\begin{equation}
    \left\{
    \begin{aligned}
    \frac{\diff x}{\diff t} = \epsilon_1 x, \\
    \frac{\diff y}{\diff t} = - \epsilon_2 y,
\end{aligned}
\right.
\end{equation}
ситуація не змінилася, точка все ще сідло. \\

Для точки $(\epsilon_1 / \beta_1, 0)$:
\begin{equation}
    \left\{
    \begin{aligned}
    \frac{\diff x}{\diff t} &= - \epsilon_1 x \left( x - \frac{\epsilon_1}{\beta_1} \right) - \frac{\gamma_1 \epsilon_1}{\beta_1} \cdot y, \\
    \frac{\diff y}{\diff t} &= \left( - \epsilon_2 + \frac{\gamma_2 \epsilon_1}{\beta_1} \right) \cdot y.
\end{aligned}
\right.
\end{equation}

Робимо заміну змінних $x - \epsilon_1 / \beta_1 = u$, тоді система набуває вигляду
\begin{equation}
    \left\{
    \begin{aligned}
    \frac{\diff u}{\diff t} &= - \epsilon_1 u - \frac{\gamma_1 \epsilon_1}{\beta_1} \cdot y, \\
    \frac{\diff y}{\diff t} &= \frac{- \epsilon_2 \beta_1 + \gamma_2 \epsilon_1}{\beta_1} \cdot y,
\end{aligned}
\right.
\end{equation}

/* обчислення характеристичного многочлена як визначника і власних чисел */ \\

звідки $\lambda_1 = - \epsilon_1 < 0$, $\lambda_2 = \dfrac{\gamma_2 \epsilon_1 - \epsilon_2 \beta_1}{\beta_1} > 0$, ще одне сідло. \\

Для точки $(\epsilon_2 / \gamma_2, (\epsilon_1 \gamma_2 - \beta_1 \epsilon_2) / (\gamma_1 \gamma_2))$:
\begin{equation}
    \left\{
    \begin{aligned}
    \frac{\diff x}{\diff t} &= \left( \epsilon_1 - \frac{\epsilon_1 \gamma_2 - \beta_1 \epsilon_2}{\gamma_2} - 2 \frac{\beta_1}{\epsilon_2}{\gamma_2} \right) \cdot (x - x^*) - \frac{\gamma_1 \epsilon_2}{\gamma_2} \cdot (y - y^*), \\
    \frac{\diff y}{\diff t} &= \frac{\epsilon_1 \gamma_2 - \beta_1 \epsilon_2}{\gamma_1} \cdot (x - x^*) .
\end{aligned}
\right.
\end{equation}

Робимо заміну змінних $x - x^* = u$, $y - y^* = v$, тоді система набуває вигляду
\begin{equation}
    \left\{
    \begin{aligned}
    \frac{\diff u}{\diff t} &= \frac{\beta_1 \epsilon_2 - 2 \beta_1 \epsilon_2}{\gamma_2} \cdot u - \frac{\gamma_1 \epsilon_2}{\gamma_2} \cdot v, \\
    \frac{\diff v}{\diff t} &= \frac{\epsilon_1 \gamma_2 - \beta_1 \epsilon_2}{\gamma_1} \cdot u,
\end{aligned}
\right.
\end{equation}

/* обчислення характеристичного многочлена як визначника і власних чисел */ \\

звідки $\lambda_{1,2} = \frac12 \left( - \frac{\epsilon_2 \beta_1}{\gamma_2} \pm \sqrt{\frac{\epsilon_2^2 \beta_1^2}{\gamma_2^2} - 4 \frac{\epsilon_1 \gamma_2 - \beta_1 \epsilon_2}{\gamma_2 \epsilon_2} } \right)$, тобто $Re \lambda_{1, 2} < 0$, стійка, тобто всі траєкторії з часом будуть збігатися до неї.

\subsection{Класична модель епідемії}

Однією з найпростіших моделей епідемії є модель Кермака-Маккендрика (1927 р.). \\

Нехай $N_1(t)$ -- частка населення, що сприймає зараження, $N_2(t)$ -- частка населення, що вже заражена, $N_3(t)$ -- частка населення, яка не сприймає зараження завдяки виробленому імунітету. Кількість людей що знову заражаються пропорційна $N_1 \cdot N_2$; якщо швидкість росту числа не сприятливих до зараження пропорційна числу заражених, то математичну модель можна записати ось в такому вигляді:
\begin{equation}
    \label{eq:2.4}
    \left\{
        \begin{aligned}
            \frac{\diff N_1}{\diff t} &= - N_1 \cdot N_2, \\
            \frac{\diff N_2}{\diff t} &= - N_2 \cdot (1 - N_1), \\
            \frac{\diff N_3}{\diff t} &= N_2.
        \end{aligned}
    \right.
\end{equation}

Для простоти коефіцієнти пропорційності покладено рівними одиниці. \\

Якщо почленно додати всі ці рівності, то отримаємо
\begin{equation}
    \dot N_1 + \dot N_2 + \dot N_3 = 0 \implies N_1 + N_2 + N_3 = const.
\end{equation}

\end{document}
