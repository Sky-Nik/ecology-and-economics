% cd ..\..\Users\NikitaSkybytskyi\Desktop\c3s2\ecology-and-economics\lab-3\tex
% cls && pdflatex -shell-escape report.tex && cls && pdflatex -shell-escape report.tex && del report.aux, report.toc, report.log, report.out && start report.pdf
\input{lab.sty}

\begin{document}

\cover{3}{Попит та пропозиція. Ринкова рівновага. \\ Стабільність рівноваги. Вплив дотації}

\tableofcontents

\section{Теоретичні відомості}



\section{Чисельне моделювання}

Було використано мову програмування \texttt{Python} і модуль \texttt{scipy}.

\subsection{Код}

\inputminted{python}{lab-3/py/all.py}
% \inputminted{python}{../py/all.py}

\subsection{Графіки}

\begin{figure}{H}
	\centering
	\includegraphics[width=\textwidth]{p_q.png}
\end{figure}

\begin{figure}{H}
	\centering
	\includegraphics[width=\textwidth]{q_p.png}
\end{figure}

\begin{figure}{H}
	\centering
	\includegraphics[width=\textwidth]{dotation.png}
\end{figure}

Як бачимо, отримані результати відповідають теоретичним очікуванням.

\end{document}